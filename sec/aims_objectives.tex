\newpage
\section{Research aims and objectives}

\subsection{Research question}
What is the relationship between allometry-derived EAGB and EC, with Landsat derived products and meteorology data within the tropical savanna of the NT?
\subsection{Aim}
This study aims to use a medium resolution remote sensing product together with meteorology data to estimate AGB and/or C across NT tropical savannas. This will be accomplished by assessing if a relationship can be identified between allometry calculated EAGB and Landsat derived products including but not limited to: SR, SI, modeled fractional cover (FC), modeled woody foliage projective cover (WFPC), fire scar mapping (FSM) and modeled canopy height (CH) etc., in conjunction with freely available meteorological data-sets using machine learning.

\subsection{Objectives}
The objectives of this study include:

\begin{enumerate}
    \item To quantify EAGB extent within NT tropical savanna.
    \item To quantify EC within NT tropical savanna.
    \item To investigate relationships between EAGB and EC with Landsat derived SR, SI and modeled biophysical metrics.
    \item To determine which Landsat product parameters or indices yield strongest correlations through principle component analysis (PCA).
    \item To determine how fire interacts with the model and mask out if required.
    \item To develop several machine learning models to predict EAGB and EC extents from Landsat derived parameters.
    \item To assess each of the models accuracy and precision.
    \item To produce a python pipeline that can apply the most precise EAGB and EC extent models to either Landsat seasonal composites or Landsat single date imagery.

\end{enumerate}